\documentclass[a4paper]{article}
\author{Narendra Syahrasyad}

\usepackage[english]{babel}
\usepackage[utf8]{inputenc}
\usepackage{amsmath}
\usepackage{amsthm}
\usepackage{graphicx}
\usepackage[colorinlistoftodos]{todonotes}
\newcommand{\code}[1]{\texttt{#1}}
\newcommand{\centerfig}[2]{\begin{center}\includegraphics[width=#1\textwidth]{#2}\end{center}}

\title{ISA and Sequential CPU\\
\vspace{5mm}
\large CPSC 313 M1}
\date{}

\begin{document}
\maketitle

\section{Processing in Stages}

Processing an instruction involves a number of different operations.

\begin{enumerate}
    \item \textbf{Fetch}
    
    Reads bytes of an instruction from memory. Parses the instruction into several variables:
    \begin{enumerate}
        \item \code{icode} - instruction code
        \item \code{ifun} - instruction function
        \item \code{rA} - first register operand, if exists
        \item \code{rB} - second register operand, if exists
        \item \code{valC} - 4 byte word included in instruction, if exists
    \end{enumerate}
    Fetch computes an additional variable \code{valP}, which is equal to \code{PC} $+$ length of fetched instruction.
    
    \item \textbf{Decode}
    
    Reads values from register file into \code{valA} and \code{valB}. Typically reads from registers \code{rA} and \code{rB}, but sometimes from other registers.
    
    \item \textbf{Execute}
    
    ALU either performs the operation specified by the instruction, computes address of a memory reference, or manipulates the stack pointer. The result is saved onto \code{valE}.
    
    Condition codes are set afterwards. Jump instructions use these codes to determine if a branch should be taken.
    
    \item \textbf{Memory}
    
    Reads/writes data from/to memory. Value read saved as \code{valM}.
    
    \item \textbf{Write back}
    
    Writes up to two results to the register file.
    
    \item \textbf{PC update}
    
    The PC is set to the address of the next instruction.
\end{enumerate}

\subsection{Condition Codes}
There are three condition codes:

\begin{itemize}
    \item \code{ZF} Zero Flag: Indicates if the result \code{valE} was zero
    \item \code{SF} Sign Flag: Indicates if the result was negative
    \item \code{OF} Overflow Flag: Indicates if the result of the operation did not fit in the register width
\end{itemize}

\subsection{Execution of OPq}
All \code{OPq} operations follow the same format:

\begin{center}
    \begin{tabular}{l l}
        Stage & Generic \\
        \hline
        Fetch & \code{icode:iFun} $\leftarrow$ $M_1$[PC]\\
            & \code{rA:rB} $\leftarrow$ $M_1$[PC + 1]\\
            & \\
            & \code{valP} $\leftarrow$ PC + 2\\
        Decode & \code{valA} $\leftarrow$ R[rA]\\
            & \code{valB} $\leftarrow$ R[rB]\\
        Execute & \code{valE} $\leftarrow$ \code{valB} OP \code{valA}\\
                & set condition codes \\
        Memory & \\
        Write back & R[rB] $\leftarrow$ \code{valE}\\
        PC update & PC $\leftarrow$ \code{valP}
    \end{tabular}
\end{center}

\subsection{Jump/Call/Ret Instructions}
In \code{Y86} ISA the stack grows \textbf{down}. 
\begin{itemize}
    \item The \code{call} instruction decrements the stack pointer and saves the return address to the first element of the stack.
    \item The \code{ret} instruction increments the stack pointer and gets the return address from the top of the stack.
\end{itemize}

\section{Hardware Structure}

\centerfig{0.8}{seq-cpu.png}

The different hardware units are associated with different processing stages.

\begin{enumerate}
    \item \textbf{Fetch}
    
    Using the \code{PC} register at the bottom, the instruction memory reads and parses the bytes of the instruction. The PC increment unit computes \code{valP}.

    \item \textbf{Decode}
    
    The register file uses its A and B ports to read the values of \code{rA} and \code{rB} to \code{valA} and \code{valB}.

    \item \textbf{Execute}
    
    The ALU computes the appropriate value according to the instruction type. The condition code register (CC) contains all condition codes. It computes \code{Cnd} which is used for all conditional instructions.

    \item \textbf{Memory}
    
    The data memory reads/writes a word of memory when executing a memory instruction.

    \item \textbf{Write back}
    
    Writes back to register file through ports E and M. Port E is for values calculated by ALU, port M is for values read from data memory.

\end{enumerate}

\end{document}